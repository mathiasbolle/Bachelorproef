%frequently used words:
\newcommand{\uiTestAut}{UI test automation}


%---------- Inleiding ---------------------------------------------------------

\section{Introductie}%
\label{sec:introductie}

% probleemstelling?
Android gebruikers komen dagelijks gemiddeld met 10 apps per dag in contact \autocite{PANKO2018} . 
Hierdoor is het belangrijk dat de applicaties op een correcte manier kunnen worden gebruikt zonder 
dat er onverwachte fouten gebeuren. Denk maar aan foute validatie, \dots{} % nog verder aanvullen
% schokkerend feit?

UI test automation helpt hierbij aangezien ze de user interface testen waar de gebruiker mee in contact komt.
Volgens \textcite{Microfocus} Het gebruik van UI automation heeft ook een aantal voordelen voor het bedrijf zelf.
Een aantal voordelen hiervan zijn:
\begin{itemize}
\item 53.4 perc 
\item nog iets

\end{itemize}

%---------- Stand van zaken ---------------------------------------------------

\section{State-of-the-art}%
\label{sec:state-of-the-art}

% Voor literatuurverwijzingen zijn er twee belangrijke commando's:
% \autocite{KEY} => (Auteur, jaartal) Gebruik dit als de naam van de auteur
%   geen onderdeel is van de zin.
% \textcite{KEY} => Auteur (jaartal)  Gebruik dit als de auteursnaam wel een
%   functie heeft in de zin (bv. ``Uit onderzoek door Doll & Hill (1954) bleek
%   ...'')


%---------- Methodologie ------------------------------------------------------
\section{Methodologie}%
\label{sec:methodologie}


%---------- Verwachte resultaten ----------------------------------------------
\section{Verwacht resultaat, conclusie}%
\label{sec:verwachte_resultaten}

